\documentclass[12pt]{article}

\usepackage[utf8]{inputenc}
\usepackage[T1]{fontenc}
\usepackage{lmodern}  % makes ligatures copy-pasteable

\usepackage[ngerman]{babel}
\usepackage{amsmath}

\title{System mit Zentralmasse}
\author{Christian Fuchs}

\begin{document}

\maketitle

\begin{abstract}
Bewegungsgleichung für ein System mit einer einzigen dominanten Zentralmasse in einem Bezugssystem, welches mit konstanter Winkelgeschwindigkeit um die Zentralmasse rotiert.
\end{abstract}

\section{Grundgleichungen}

Beschleunigung eines Massepunktes $\vec{x} = \left( \begin{matrix} x & y & z \end{matrix} \right)$ um eine unbewegliche Zentralmasse $M$ im Koordinatenursprung $\vec{0}$ mit Gravitationskonstante $\gamma$:

\begin{equation}
\vec{\ddot{x}} = - \frac{\vec{x}}{\left| \vec{x} \right|^3} \gamma M \label{eq:basic}
\end{equation}

Beziehung zwischen zueinander rotierenden Systemen $\vec{x}$ und $\vec{x}'$ mit gemeinsamem Ursprung $\vec{0} = \vec{0}'$ und mit Rotationsvektor $\vec{\omega} = \left( \begin{matrix} 0 & 0 & \omega \end{matrix} \right)$:

\begin{eqnarray}
\mathcal{D} &=& \left(
\begin{matrix}
  \cos \omega t & -\sin \omega t & 0 \\
  \sin \omega t &  \cos \omega t & 0 \\
  0             &  0             & 1
\end{matrix}
\right) \nonumber \\
\vec{x}  &=& \mathcal{D} \vec{x}' \label{eq:xtoxprime}
\end{eqnarray}

\noindent und umgekehrt mit dem Rotationsvektor $\vec{\omega}' = \left( \begin{matrix} 0 & 0 & -\omega \end{matrix} \right)$:

\begin{eqnarray}
\vec{x}'  &=& \mathcal{D}^{-1} \vec{x} \nonumber \\
\mathcal{D}^{-1} &=& \left(
\begin{matrix}
  \cos \omega t &  \sin \omega t & 0 \\
 -\sin \omega t &  \cos \omega t & 0 \\
  0             &  0             & 1
\end{matrix}
\right) \nonumber
\end{eqnarray}

\section{Übertragung ins rotierende System}

Die Gleichheit $\left| \vec{x} \right| = \left| \vec{x}' \right|$ ergibt sich wegen Gleichung~\ref{eq:xtoxprime} und wird nicht näher beschrieben.

Die linke Seite von Gleichung~\ref{eq:basic} ändert sich mit Gleichung~\ref{eq:xtoxprime} wie folgt:

\begin{eqnarray}
\vec{\ddot{x}}
  &=& \frac{\partial^2}{\partial t^2} \vec{x} \nonumber \\
  &=& \frac{\partial}{\partial t} \frac{\partial}{\partial t} \vec{x} \nonumber \\
  &=& \frac{\partial}{\partial t} \frac{\partial}{\partial t} \left( \mathcal{D} \vec{x}' \right) \nonumber \\
  &=& \frac{\partial}{\partial t} \left( \frac{\partial\mathcal{D}}{\partial t} \vec{x}' + \mathcal{D} \frac{\partial\vec{x}'}{\partial t} \right) \nonumber \\
  &=& \frac{\partial^2\mathcal{D}}{\partial t^2} \vec{x}'
     + 2 \frac{\partial\mathcal{D}}{\partial t} \frac{\partial\vec{x}'}{\partial t}
     + \mathcal{D} \frac{\partial^2\vec{x}'}{\partial t^2} \nonumber \\
  &=& \ddot{\mathcal{D}}\vec{x}' + 2 \dot{\mathcal{D}}\dot{\vec{x}}' + \mathcal{D}\ddot{\vec{x}}' \nonumber 
\end{eqnarray}

Die Zeitableitungen von $\mathcal{D}$ sind:

\begin{eqnarray}
\dot{\mathcal{D}} &=& \omega \left(
\begin{matrix}
 -\sin \omega t & -\cos \omega t & 0 \\
  \cos \omega t & -\sin \omega t & 0 \\
  0             &  0             & 0
\end{matrix}
\right) \nonumber \\
\ddot{\mathcal{D}} &=& -\omega^2 \mathcal{D} \nonumber 
\end{eqnarray}

Damit und mit der Ersetzung verbleibender $\vec{x}$ durch $\mathcal{D}\vec{x}'$ ergibt sich als Bewegungsgleichung im Bezugssystem $\vec{x}'$:

\begin{eqnarray}
\vec{\ddot{x}} &=& - \frac{\vec{x}}{\left| \vec{x} \right|^3} \gamma M \nonumber \\
\ddot{\mathcal{D}}\vec{x}' + 2 \dot{\mathcal{D}}\dot{\vec{x}}' + \mathcal{D}\ddot{\vec{x}}' &=& - \mathcal{D}\frac{\vec{x}'}{\left| \vec{x}' \right|^3} \gamma M \nonumber \\
-\omega^2\mathcal{D}\vec{x}' + 2 \dot{\mathcal{D}}\dot{\vec{x}}' + \mathcal{D}\ddot{\vec{x}}' &=& - \mathcal{D}\frac{\vec{x}'}{\left| \vec{x}' \right|^3} \gamma M \nonumber \\ 
\mathcal{D}\ddot{\vec{x}}' &=& - \mathcal{D}\frac{\vec{x}'}{\left| \vec{x}' \right|^3} \gamma M + \omega^2\mathcal{D}\vec{x}' - 2 \dot{\mathcal{D}}\dot{\vec{x}}' \nonumber \\
\ddot{\vec{x}}' &=& - \frac{\vec{x}'}{\left| \vec{x}' \right|^3} \gamma M + \omega^2\vec{x} - 2 \left( \mathcal{D}^{-1} \dot{\mathcal{D}}\right)\dot{\vec{x}}' \nonumber \\
\ddot{\vec{x}} &=& - \frac{\vec{x}'}{\left| \vec{x}' \right|^3} \gamma M + \omega^2\vec{x}' - 2 \left( \vec{\omega} \times \dot{\vec{x}}'\right) \label{eq:translated}
\end{eqnarray}

In den letzten beiden Zeilen wurde von links mit $\mathcal{D}^{-1}$ multipliziert und ausgenutzt, dass das Punktprodukt $\mathcal{D}^{-1}\dot{\mathcal{D}} \cdot \vec{v}$ gleich dem Kreuzprodukt $\vec{\omega}\times\vec{v}$ ist.

$\omega^2\vec{x}'$ ist die Beschleunigung durch die Fliehkraft und $- 2 \left( \vec{\omega} \times \dot{\vec{x}}'\right)$ die Beschleuningung durch die Corioliskraft.

\end{document}
