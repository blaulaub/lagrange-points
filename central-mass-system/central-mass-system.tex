\documentclass[12pt]{article}

\usepackage[utf8]{inputenc}
\usepackage[T1]{fontenc}
\usepackage{lmodern}  % makes ligatures copy-pasteable

\usepackage[ngerman]{babel}
\usepackage{amsmath}

\title{System mit Zentralmasse}
\author{Christian Fuchs}

\begin{document}

\maketitle

\begin{abstract}
Bewegungsgleichung für ein System mit einer einzigen dominanten Zentralmasse in einem Bezugssystem, welches mit konstanter Winkelgeschwindigkeit um die Zentralmasse rotiert.
\end{abstract}

\tableofcontents

\section{Grundgleichungen}

Beschleunigung eines Massepunktes um eine unbewegliche Zentralmasse $M$ im Koordinatenursprung $\vec{0}$ mit Gravitationskonstante $\gamma$:

\begin{equation}
\vec{\ddot{x}} = - \frac{\vec{x}}{\left| \vec{x} \right|^3} \gamma M
\end{equation}

Beziehung zwischen zueinander rotierenden Systemen $\vec{x}$ und $\vec{x'}$ mit gemeinsamem Ursprung $\vec{0} = \vec{0'}$ und mit Rotationsvektor $\vec{\omega} = \left( \begin{matrix} 0 & 0 & \omega \end{matrix} \right)$:

\begin{eqnarray}
\mathcal{D} &= \left(
\begin{matrix}
  \cos \omega t & \sin \omega t & 0 \\
 -\sin \omega t & \cos \omega t & 0 \\
  0             & 0             & 1
\end{matrix}
\right)
\end{eqnarray}



\end{document}
